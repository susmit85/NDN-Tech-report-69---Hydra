\subsection{Design Approaches} \label{subsec:design-approach}

\begin{itemize}
\item For all new and existing data under our control: adopt the principle in naming.
For all existing data using URL as unique names: gradually remove the ``location" semantic in those names, treating them simply as unique data identifiers.

\item The user interfaces should be simple and based on file names. 
 
    Operations such as insertion or deletion should require the user to send an Interest into the network (anycast). The Interest then brings back a success or failure response (along with diagnosis codes, if necessary). The users and/or applications should not be exposed to the internal workings of Hydra, such as pub-sub models, replication details, and location information of the nodes. 
    
    Users do not need to know the location of the nodes. NDN routing allows users to reach the ``nearest" node. However, underlying applications might need to utilize some of the location information occasionally (e.g., forwarding hint to a specific node for data retrieval).
    % \item 
    % \item Hydra design aims to refrain from using Interest packets for carrying Data. We will use application parameters to carry required parameters for interactions.
\end{itemize}

We prefer decentralized approaches in Hydra design (when possible) even though there may be an extra cost of decentralization - communication overhead.